%%
% Copyright (c) 2017 - 2025, Pascal Wagler;
% Copyright (c) 2014 - 2025, John MacFarlane
%
% All rights reserved.
%
% Redistribution and use in source and binary forms, with or without
% modification, are permitted provided that the following conditions
% are met:
%
% - Redistributions of source code must retain the above copyright
% notice, this list of conditions and the following disclaimer.
%
% - Redistributions in binary form must reproduce the above copyright
% notice, this list of conditions and the following disclaimer in the
% documentation and/or other materials provided with the distribution.
%
% - Neither the name of John MacFarlane nor the names of other
% contributors may be used to endorse or promote products derived
% from this software without specific prior written permission.
%
% THIS SOFTWARE IS PROVIDED BY THE COPYRIGHT HOLDERS AND CONTRIBUTORS
% "AS IS" AND ANY EXPRESS OR IMPLIED WARRANTIES, INCLUDING, BUT NOT
% LIMITED TO, THE IMPLIED WARRANTIES OF MERCHANTABILITY AND FITNESS
% FOR A PARTICULAR PURPOSE ARE DISCLAIMED. IN NO EVENT SHALL THE
% COPYRIGHT OWNER OR CONTRIBUTORS BE LIABLE FOR ANY DIRECT, INDIRECT,
% INCIDENTAL, SPECIAL, EXEMPLARY, OR CONSEQUENTIAL DAMAGES (INCLUDING,
% BUT NOT LIMITED TO, PROCUREMENT OF SUBSTITUTE GOODS OR SERVICES;
% LOSS OF USE, DATA, OR PROFITS; OR BUSINESS INTERRUPTION) HOWEVER
% CAUSED AND ON ANY THEORY OF LIABILITY, WHETHER IN CONTRACT, STRICT
% LIABILITY, OR TORT (INCLUDING NEGLIGENCE OR OTHERWISE) ARISING IN
% ANY WAY OUT OF THE USE OF THIS SOFTWARE, EVEN IF ADVISED OF THE
% POSSIBILITY OF SUCH DAMAGE.
%%

%%
% This is the Eisvogel pandoc LaTeX template.
%
% For usage information and examples visit the official GitHub page:
% https://github.com/Wandmalfarbe/pandoc-latex-template
%%
% Options for packages loaded elsewhere
\PassOptionsToPackage{unicode}{hyperref}
\PassOptionsToPackage{hyphens}{url}
\PassOptionsToPackage{dvipsnames,svgnames,x11names,table}{xcolor}
\documentclass[
  polish,
  paper=a4,
  ,captions=tableheading
]{scrartcl}
\usepackage{xcolor}
\usepackage[margin=2.5cm,includehead=true,includefoot=true,centering,]{geometry}
\usepackage{amsmath,amssymb}


% add backlinks to footnote references, cf. https://tex.stackexchange.com/questions/302266/make-footnote-clickable-both-ways
\usepackage{footnotebackref}
\setcounter{secnumdepth}{-\maxdimen} % remove section numbering
\usepackage{iftex}
\ifPDFTeX
  \usepackage[T1]{fontenc}
  \usepackage[utf8]{inputenc}
  \usepackage{textcomp} % provide euro and other symbols
\else % if luatex or xetex
  \usepackage{unicode-math} % this also loads fontspec
  \defaultfontfeatures{Scale=MatchLowercase}
  \defaultfontfeatures[\rmfamily]{Ligatures=TeX,Scale=1}
\fi
\usepackage{lmodern}
\ifPDFTeX\else
  % xetex/luatex font selection
\fi
% Use upquote if available, for straight quotes in verbatim environments
\IfFileExists{upquote.sty}{\usepackage{upquote}}{}
\IfFileExists{microtype.sty}{% use microtype if available
  \usepackage[]{microtype}
  \UseMicrotypeSet[protrusion]{basicmath} % disable protrusion for tt fonts
}{}

% Use setspace anyway because we change the default line spacing.
% The spacing is changed early to affect the titlepage and the TOC.
\usepackage{setspace}
\setstretch{1.2}
\makeatletter
\@ifundefined{KOMAClassName}{% if non-KOMA class
  \IfFileExists{parskip.sty}{%
    \usepackage{parskip}
  }{% else
    \setlength{\parindent}{0pt}
    \setlength{\parskip}{6pt plus 2pt minus 1pt}}
}{% if KOMA class
  \KOMAoptions{parskip=half}}
\makeatother
\usepackage{listings}
\newcommand{\passthrough}[1]{#1}
\lstset{defaultdialect=[5.3]Lua}
\lstset{defaultdialect=[x86masm]Assembler}
\usepackage{etoolbox}
\BeforeBeginEnvironment{lstlisting}{\par\noindent\begin{minipage}{\linewidth}}
\AfterEndEnvironment{lstlisting}{\end{minipage}\par\addvspace{\topskip}}
\usepackage{color}
\usepackage{fancyvrb}
\newcommand{\VerbBar}{|}
\newcommand{\VERB}{\Verb[commandchars=\\\{\}]}
\DefineVerbatimEnvironment{Highlighting}{Verbatim}{commandchars=\\\{\}}
% Add ',fontsize=\small' for more characters per line
\usepackage{framed}
\definecolor{shadecolor}{RGB}{248,248,248}
\newenvironment{Shaded}{\begin{snugshade}}{\end{snugshade}}
\newcommand{\AlertTok}[1]{\textcolor[rgb]{0.94,0.16,0.16}{#1}}
\newcommand{\AnnotationTok}[1]{\textcolor[rgb]{0.56,0.35,0.01}{\textbf{\textit{#1}}}}
\newcommand{\AttributeTok}[1]{\textcolor[rgb]{0.13,0.29,0.53}{#1}}
\newcommand{\BaseNTok}[1]{\textcolor[rgb]{0.00,0.00,0.81}{#1}}
\newcommand{\BuiltInTok}[1]{#1}
\newcommand{\CharTok}[1]{\textcolor[rgb]{0.31,0.60,0.02}{#1}}
\newcommand{\CommentTok}[1]{\textcolor[rgb]{0.56,0.35,0.01}{\textit{#1}}}
\newcommand{\CommentVarTok}[1]{\textcolor[rgb]{0.56,0.35,0.01}{\textbf{\textit{#1}}}}
\newcommand{\ConstantTok}[1]{\textcolor[rgb]{0.56,0.35,0.01}{#1}}
\newcommand{\ControlFlowTok}[1]{\textcolor[rgb]{0.13,0.29,0.53}{\textbf{#1}}}
\newcommand{\DataTypeTok}[1]{\textcolor[rgb]{0.13,0.29,0.53}{#1}}
\newcommand{\DecValTok}[1]{\textcolor[rgb]{0.00,0.00,0.81}{#1}}
\newcommand{\DocumentationTok}[1]{\textcolor[rgb]{0.56,0.35,0.01}{\textbf{\textit{#1}}}}
\newcommand{\ErrorTok}[1]{\textcolor[rgb]{0.64,0.00,0.00}{\textbf{#1}}}
\newcommand{\ExtensionTok}[1]{#1}
\newcommand{\FloatTok}[1]{\textcolor[rgb]{0.00,0.00,0.81}{#1}}
\newcommand{\FunctionTok}[1]{\textcolor[rgb]{0.13,0.29,0.53}{\textbf{#1}}}
\newcommand{\ImportTok}[1]{#1}
\newcommand{\InformationTok}[1]{\textcolor[rgb]{0.56,0.35,0.01}{\textbf{\textit{#1}}}}
\newcommand{\KeywordTok}[1]{\textcolor[rgb]{0.13,0.29,0.53}{\textbf{#1}}}
\newcommand{\NormalTok}[1]{#1}
\newcommand{\OperatorTok}[1]{\textcolor[rgb]{0.81,0.36,0.00}{\textbf{#1}}}
\newcommand{\OtherTok}[1]{\textcolor[rgb]{0.56,0.35,0.01}{#1}}
\newcommand{\PreprocessorTok}[1]{\textcolor[rgb]{0.56,0.35,0.01}{\textit{#1}}}
\newcommand{\RegionMarkerTok}[1]{#1}
\newcommand{\SpecialCharTok}[1]{\textcolor[rgb]{0.81,0.36,0.00}{\textbf{#1}}}
\newcommand{\SpecialStringTok}[1]{\textcolor[rgb]{0.31,0.60,0.02}{#1}}
\newcommand{\StringTok}[1]{\textcolor[rgb]{0.31,0.60,0.02}{#1}}
\newcommand{\VariableTok}[1]{\textcolor[rgb]{0.00,0.00,0.00}{#1}}
\newcommand{\VerbatimStringTok}[1]{\textcolor[rgb]{0.31,0.60,0.02}{#1}}
\newcommand{\WarningTok}[1]{\textcolor[rgb]{0.56,0.35,0.01}{\textbf{\textit{#1}}}}

% Workaround/bugfix from jannick0.
% See https://github.com/jgm/pandoc/issues/4302#issuecomment-360669013)
% or https://github.com/Wandmalfarbe/pandoc-latex-template/issues/2
%
% Redefine the verbatim environment 'Highlighting' to break long lines (with
% the help of fvextra). Redefinition is necessary because it is unlikely that
% pandoc includes fvextra in the default template.
\usepackage{fvextra}
\DefineVerbatimEnvironment{Highlighting}{Verbatim}{breaklines,fontsize=\small,commandchars=\\\{\}}

\usepackage{graphicx}
\makeatletter
\newsavebox\pandoc@box
\newcommand*\pandocbounded[1]{% scales image to fit in text height/width
  \sbox\pandoc@box{#1}%
  \Gscale@div\@tempa{\textheight}{\dimexpr\ht\pandoc@box+\dp\pandoc@box\relax}%
  \Gscale@div\@tempb{\linewidth}{\wd\pandoc@box}%
  \ifdim\@tempb\p@<\@tempa\p@\let\@tempa\@tempb\fi% select the smaller of both
  \ifdim\@tempa\p@<\p@\scalebox{\@tempa}{\usebox\pandoc@box}%
  \else\usebox{\pandoc@box}%
  \fi%
}
% Set default figure placement to htbp
% Make use of float-package and set default placement for figures to H.
% The option H means 'PUT IT HERE' (as  opposed to the standard h option which means 'You may put it here if you like').
\usepackage{float}
\floatplacement{figure}{H}
\makeatother
\ifLuaTeX
\usepackage[bidi=basic,shorthands=off,]{babel}
\else
\usepackage[bidi=default,shorthands=off,]{babel}
\fi
\ifLuaTeX
  \usepackage{selnolig} % disable illegal ligatures
\fi
\setlength{\emergencystretch}{3em} % prevent overfull lines
\providecommand{\tightlist}{%
  \setlength{\itemsep}{0pt}\setlength{\parskip}{0pt}}
\usepackage{bookmark}
\IfFileExists{xurl.sty}{\usepackage{xurl}}{} % add URL line breaks if available
\urlstyle{same}
\definecolor{default-linkcolor}{HTML}{A50000}
\definecolor{default-filecolor}{HTML}{A50000}
\definecolor{default-citecolor}{HTML}{4077C0}
\definecolor{default-urlcolor}{HTML}{4077C0}

\hypersetup{
  pdftitle={Konfiguracja Ubuntu Server 24.04},
  pdfauthor={Rafał Kiepiela},
  pdflang={pl},
  hidelinks,
  breaklinks=true,
  pdfcreator={LaTeX via pandoc with the Eisvogel template}}

\title{Konfiguracja Ubuntu Server 24.04}
\author{Rafał Kiepiela}
\date{}


%
% for the background color of the title page
%
\usepackage{pagecolor}
\usepackage{afterpage}
\usepackage{tikz}
\usepackage[margin=2.5cm,includehead=true,includefoot=true,centering]{geometry}

%
% break urls
%
\PassOptionsToPackage{hyphens}{url}

%
% When using babel or polyglossia with biblatex, loading csquotes is recommended
% to ensure that quoted texts are typeset according to the rules of your main language.
%
\usepackage{csquotes}

%
% captions
%
\definecolor{caption-color}{HTML}{777777}
\usepackage[font={stretch=1.2}, textfont={color=caption-color}, position=top, skip=4mm, labelfont=bf, singlelinecheck=false, justification=raggedright]{caption}
\setcapindent{0em}

%
% blockquote
%
\definecolor{blockquote-border}{RGB}{221,221,221}
\definecolor{blockquote-text}{RGB}{119,119,119}
\usepackage{mdframed}
\newmdenv[rightline=false,bottomline=false,topline=false,linewidth=3pt,linecolor=blockquote-border,skipabove=\parskip]{customblockquote}
\renewenvironment{quote}{\begin{customblockquote}\list{}{\rightmargin=0em\leftmargin=0em}%
\item\relax\color{blockquote-text}\ignorespaces}{\unskip\unskip\endlist\end{customblockquote}}

%
% Source Sans Pro as the default font family
% Source Code Pro for monospace text
%
% 'default' option sets the default
% font family to Source Sans Pro, not \sfdefault.
%
\ifnum 0\ifxetex 1\fi\ifluatex 1\fi=0 % if pdftex
    \usepackage[default]{sourcesanspro}
  \usepackage{sourcecodepro}
  \else % if not pdftex
    \usepackage[default]{sourcesanspro}
  \usepackage{sourcecodepro}

  % XeLaTeX specific adjustments for straight quotes: https://tex.stackexchange.com/a/354887
  % This issue is already fixed (see https://github.com/silkeh/latex-sourcecodepro/pull/5) but the
  % fix is still unreleased.
  % TODO: Remove this workaround when the new version of sourcecodepro is released on CTAN.
  \ifxetex
    \makeatletter
    \defaultfontfeatures[\ttfamily]
      { Numbers   = \sourcecodepro@figurestyle,
        Scale     = \SourceCodePro@scale,
        Extension = .otf }
    \setmonofont
      [ UprightFont    = *-\sourcecodepro@regstyle,
        ItalicFont     = *-\sourcecodepro@regstyle It,
        BoldFont       = *-\sourcecodepro@boldstyle,
        BoldItalicFont = *-\sourcecodepro@boldstyle It ]
      {SourceCodePro}
    \makeatother
  \fi
  \fi

%
% heading color
%
\definecolor{heading-color}{RGB}{40,40,40}
\addtokomafont{section}{\color{heading-color}}
% When using the classes report, scrreprt, book,
% scrbook or memoir, uncomment the following line.
%\addtokomafont{chapter}{\color{heading-color}}

%
% variables for title, author and date
%
\usepackage{titling}
\title{Konfiguracja Ubuntu Server 24.04}
\author{Rafał Kiepiela}
\date{}

%
% tables
%

%
% remove paragraph indentation
%
\setlength{\parindent}{0pt}
\setlength{\parskip}{6pt plus 2pt minus 1pt}
\setlength{\emergencystretch}{3em}  % prevent overfull lines

%
%
% Listings
%
%


%
% general listing colors
%
\definecolor{listing-background}{HTML}{F7F7F7}
\definecolor{listing-rule}{HTML}{B3B2B3}
\definecolor{listing-numbers}{HTML}{B3B2B3}
\definecolor{listing-text-color}{HTML}{000000}
\definecolor{listing-keyword}{HTML}{435489}
\definecolor{listing-keyword-2}{HTML}{1284CA} % additional keywords
\definecolor{listing-keyword-3}{HTML}{9137CB} % additional keywords
\definecolor{listing-identifier}{HTML}{435489}
\definecolor{listing-string}{HTML}{00999A}
\definecolor{listing-comment}{HTML}{8E8E8E}

\lstdefinestyle{eisvogel_listing_style}{
  language         = java,
  numbers          = left,
  xleftmargin      = 2.7em,
  framexleftmargin = 2.5em,
  backgroundcolor  = \color{listing-background},
  basicstyle       = \color{listing-text-color}\linespread{1.0}%
                      \lst@ifdisplaystyle%
                      \small%
                      \fi\ttfamily{},
  breaklines       = true,
  frame            = single,
  framesep         = 0.19em,
  rulecolor        = \color{listing-rule},
  frameround       = ffff,
  tabsize          = 4,
  numberstyle      = \color{listing-numbers},
  aboveskip        = 1.0em,
  belowskip        = 0.1em,
  abovecaptionskip = 0em,
  belowcaptionskip = 1.0em,
  keywordstyle     = {\color{listing-keyword}\bfseries},
  keywordstyle     = {[2]\color{listing-keyword-2}\bfseries},
  keywordstyle     = {[3]\color{listing-keyword-3}\bfseries\itshape},
  sensitive        = true,
  identifierstyle  = \color{listing-identifier},
  commentstyle     = \color{listing-comment},
  stringstyle      = \color{listing-string},
  showstringspaces = false,
  escapeinside     = {/*@}{@*/}, % Allow LaTeX inside these special comments
  literate         =
  {á}{{\'a}}1 {é}{{\'e}}1 {í}{{\'i}}1 {ó}{{\'o}}1 {ú}{{\'u}}1
  {Á}{{\'A}}1 {É}{{\'E}}1 {Í}{{\'I}}1 {Ó}{{\'O}}1 {Ú}{{\'U}}1
  {à}{{\`a}}1 {è}{{\`e}}1 {ì}{{\`i}}1 {ò}{{\`o}}1 {ù}{{\`u}}1
  {À}{{\`A}}1 {È}{{\`E}}1 {Ì}{{\`I}}1 {Ò}{{\`O}}1 {Ù}{{\`U}}1
  {ä}{{\"a}}1 {ë}{{\"e}}1 {ï}{{\"i}}1 {ö}{{\"o}}1 {ü}{{\"u}}1
  {Ä}{{\"A}}1 {Ë}{{\"E}}1 {Ï}{{\"I}}1 {Ö}{{\"O}}1 {Ü}{{\"U}}1
  {â}{{\^a}}1 {ê}{{\^e}}1 {î}{{\^i}}1 {ô}{{\^o}}1 {û}{{\^u}}1
  {Â}{{\^A}}1 {Ê}{{\^E}}1 {Î}{{\^I}}1 {Ô}{{\^O}}1 {Û}{{\^U}}1
  {œ}{{\oe}}1 {Œ}{{\OE}}1 {æ}{{\ae}}1 {Æ}{{\AE}}1 {ß}{{\ss}}1
  {ç}{{\c c}}1 {Ç}{{\c C}}1 {ø}{{\o}}1 {å}{{\r a}}1 {Å}{{\r A}}1
  {€}{{\EUR}}1 {£}{{\pounds}}1 {«}{{\guillemotleft}}1
  {»}{{\guillemotright}}1 {ñ}{{\~n}}1 {Ñ}{{\~N}}1 {¿}{{?`}}1
  {…}{{\ldots}}1 {≥}{{>=}}1 {≤}{{<=}}1 {„}{{\glqq}}1 {“}{{\grqq}}1
  {”}{{''}}1
}
\lstset{style=eisvogel_listing_style}

%
% Java (Java SE 12, 2019-06-22)
%
\lstdefinelanguage{Java}{
  morekeywords={
    % normal keywords (without data types)
    abstract,assert,break,case,catch,class,continue,default,
    do,else,enum,exports,extends,final,finally,for,if,implements,
    import,instanceof,interface,module,native,new,package,private,
    protected,public,requires,return,static,strictfp,super,switch,
    synchronized,this,throw,throws,transient,try,volatile,while,
    % var is an identifier
    var
  },
  morekeywords={[2] % data types
    % primitive data types
    boolean,byte,char,double,float,int,long,short,
    % String
    String,
    % primitive wrapper types
    Boolean,Byte,Character,Double,Float,Integer,Long,Short
    % number types
    Number,AtomicInteger,AtomicLong,BigDecimal,BigInteger,DoubleAccumulator,DoubleAdder,LongAccumulator,LongAdder,Short,
    % other
    Object,Void,void
  },
  morekeywords={[3] % literals
    % reserved words for literal values
    null,true,false,
  },
  sensitive,
  morecomment  = [l]//,
  morecomment  = [s]{/*}{*/},
  morecomment  = [s]{/**}{*/},
  morestring   = [b]",
  morestring   = [b]',
}

\lstdefinelanguage{XML}{
  morestring      = [b]",
  moredelim       = [s][\bfseries\color{listing-keyword}]{<}{\ },
  moredelim       = [s][\bfseries\color{listing-keyword}]{</}{>},
  moredelim       = [l][\bfseries\color{listing-keyword}]{/>},
  moredelim       = [l][\bfseries\color{listing-keyword}]{>},
  morecomment     = [s]{<?}{?>},
  morecomment     = [s]{<!--}{-->},
  commentstyle    = \color{listing-comment},
  stringstyle     = \color{listing-string},
  identifierstyle = \color{listing-identifier}
}

%
% header and footer
%
\usepackage[headsepline,footsepline]{scrlayer-scrpage}

\newpairofpagestyles{eisvogel-header-footer}{
  \clearpairofpagestyles
  \ihead*{Konfiguracja Ubuntu Server 24.04}
  \chead*{}
  \ohead*{}
  \ifoot*{Rafał Kiepiela}
  \cfoot*{}
  \ofoot*{\thepage}
  \addtokomafont{pageheadfoot}{\upshape}
}
\pagestyle{eisvogel-header-footer}



%
% Define watermark
%

\begin{document}

\begin{titlepage}
\newgeometry{top=2cm, right=4cm, bottom=3cm, left=4cm}
\tikz[remember picture,overlay] \node[inner sep=0pt] at (current page.center){\includegraphics[width=\paperwidth,height=\paperheight]{background2.pdf}};
\newcommand{\colorRule}[3][black]{\textcolor[HTML]{#1}{\rule{#2}{#3}}}
\begin{flushleft}
\noindent
\\[-1em]
\color[HTML]{5F5F5F}
\makebox[0pt][l]{\colorRule[435488]{1.3\textwidth}{4pt}}
\par
\noindent

% The titlepage with a background image has other text spacing and text size
{
  \setstretch{2}
  \vfill
  \vskip -8em
  \noindent {\huge \textbf{\textsf{Konfiguracja Ubuntu Server 24.04}}}
    \vskip 2em
  \noindent {\Large \textsf{Rafał Kiepiela} \vskip 0.6em \textsf{}}
  \noindent {\Large \textsf{github.com/TexturedPolak} \vskip 0.6em \textsf{}}
  \vfill
}


\end{flushleft}
\end{titlepage}
\restoregeometry
\pagenumbering{arabic}

% don't generate the default title
% \maketitle


\section{Konfiguracja Ubuntu Server
24.04}\label{konfiguracja-ubuntu-server-24.04}

\subsection{0. Słowem Wstępu}\label{sux142owem-wstux119pu}

Dokument ten przedstawia przykładową konfigurację Ubuntu Server 24.04.
Konfigurowane usługi to:

\begin{itemize}
\tightlist
\item
  sieć (netplan)
\item
  serwer DHCP (isc-dhcp-server)
\item
  serwer DNS (bind9)
\item
  routing (iptables)
\item
  serwer plików FTP (vsftpd)
\item
  serwer plików Samba (samba)
\item
  serwer WWW (apache2)
\item
  serwer SSH (openssh-server)
\item
  serwer telnet (telnetd)
\end{itemize}

Autor zaznacza, że zrzuty ekrany były kadrowane tylko i wyłącznie dla
czytelności dokumentu oraz, że jest świadomy, iż na egzaminie z
kwalifikacji INF.02 nie można kadrować zrzutów ekranów.

\subsection{1. Konfiguracja sieci}\label{konfiguracja-sieci}

Przyjęte założenia:

\begin{itemize}
\tightlist
\item
  automatycznie przydzielany adres IP (DHCP) na pierwszej karcie
  sieciowej
\item
  statyczny adres IP klasy B (w moim przypadku 172.16.0.1), maska
  podsieci 16 bitowa na drugiej karcie sieciowej
\end{itemize}

\subsubsection{Krok 1 - Sprawdzenie stanu i nazw kart
sieciowych}\label{krok-1---sprawdzenie-stanu-i-nazw-kart-sieciowych}

\begin{Shaded}
\begin{Highlighting}[]
\ExtensionTok{ip}\NormalTok{ a}
\end{Highlighting}
\end{Shaded}

\pandocbounded{\includegraphics[keepaspectratio]{/run/media/rafal/ECD5-E984/photos/ip-a-przed.png}}

Widzimy dwie karty sieciowe:

\begin{itemize}
\tightlist
\item
  enp0s3 - WAN (pierwsza karta sieciowa)
\item
  enp0s8 - LAN (druga karta sieciowa)
\end{itemize}

\subsubsection{Krok 2 - Edycja pliku konfiguracyjnego
netplan'a}\label{krok-2---edycja-pliku-konfiguracyjnego-netplana}

Nazwa pliku konfiguracyjnego może nieznacznie się różnić w zależności od
edycji systemu, dlatego zastosowano znak ``*``:

\begin{Shaded}
\begin{Highlighting}[]
\FunctionTok{sudo}\NormalTok{ nano /etc/netplan/}\PreprocessorTok{*}\NormalTok{.yaml}
\end{Highlighting}
\end{Shaded}

Edytujemy plik do pożądanego rezultatu:

\pandocbounded{\includegraphics[keepaspectratio]{/run/media/rafal/ECD5-E984/photos/nano-netplan.png}}

Omówię teraz tą konfigurację:

\begin{Shaded}
\begin{Highlighting}[]
\FunctionTok{network}\KeywordTok{:}
\AttributeTok{  }\FunctionTok{version}\KeywordTok{:}\AttributeTok{ }\DecValTok{2}
\AttributeTok{  }\FunctionTok{ethernets}\KeywordTok{:}
\CommentTok{    \# Pierwsza karta sieciowa}
\AttributeTok{    }\FunctionTok{enp0s3}\KeywordTok{:}\AttributeTok{ }
\CommentTok{      \# Automatyczne przydzielanie adresu}
\AttributeTok{      }\FunctionTok{dhcp4}\KeywordTok{:}\AttributeTok{ }\CharTok{true}\AttributeTok{ }
\CommentTok{    \# Druga karta sieciowa}
\AttributeTok{    }\FunctionTok{enp0s8}\KeywordTok{:}
\CommentTok{      \# Wyłączenie automatycznego przydzielania adresów}
\AttributeTok{      }\FunctionTok{dhcp4}\KeywordTok{:}\AttributeTok{ }\CharTok{false}\AttributeTok{ }
\CommentTok{      \# Ustawienie statycznego adresu IP (172.16.0.1, maska 16 bitowa)}
\AttributeTok{      }\FunctionTok{addresses}\KeywordTok{:}\AttributeTok{ }\KeywordTok{[}\AttributeTok{172.16.0.1/}\DecValTok{16}\KeywordTok{]}
\CommentTok{      \# Ustawianie adresów serwerów DNS}
\AttributeTok{      }\FunctionTok{nameservers}\KeywordTok{:}\AttributeTok{ }
\CommentTok{        \# Adresy serwerów DNS}
\AttributeTok{        }\FunctionTok{addresses}\KeywordTok{:}\AttributeTok{ }\KeywordTok{[}\FloatTok{172.16.0.1}\KeywordTok{,}\AttributeTok{ }\FloatTok{1.1.1.1}\KeywordTok{]}\AttributeTok{ }
\end{Highlighting}
\end{Shaded}

Należy zwrócić uwagę na zachowanie wcięć.

\subsubsection{Krok 3 - Zastosowanie konfiguracji oraz sprawdzenie
poprawności wykonanej
konfiguracji}\label{krok-3---zastosowanie-konfiguracji-oraz-sprawdzenie-poprawnoux15bci-wykonanej-konfiguracji}

\begin{Shaded}
\begin{Highlighting}[]
\FunctionTok{sudo}\NormalTok{ netplan apply}
\end{Highlighting}
\end{Shaded}

\begin{Shaded}
\begin{Highlighting}[]
\ExtensionTok{ip}\NormalTok{ a}
\end{Highlighting}
\end{Shaded}

\pandocbounded{\includegraphics[keepaspectratio]{/run/media/rafal/ECD5-E984/photos/netplan-apply.png}}

Żaden błąd nie wyskoczył, a konfiguracja sieci wygląda poprawnie.

\subsection{2. Konfiguracja serwera
DHCP}\label{konfiguracja-serwera-dhcp}

\subsubsection{Krok 0 - Instalacja pakietu
isc-dhcp-server}\label{krok-0---instalacja-pakietu-isc-dhcp-server}

\begin{Shaded}
\begin{Highlighting}[]
\FunctionTok{sudo}\NormalTok{ apt update}
\end{Highlighting}
\end{Shaded}

\begin{Shaded}
\begin{Highlighting}[]
\FunctionTok{sudo}\NormalTok{ apt install isc{-}dhcp{-}server}
\end{Highlighting}
\end{Shaded}

Ja już ten pakiet zainstalowałem szybciej, przejdźmy do konfiguracji.

\subsubsection{Krok 1 - Edycja pliku konfiguracyjnego
isc-dhcp-server}\label{krok-1---edycja-pliku-konfiguracyjnego-isc-dhcp-server}

\begin{Shaded}
\begin{Highlighting}[]
\FunctionTok{sudo}\NormalTok{ nano /etc/default/isc{-}dhcp{-}server}
\end{Highlighting}
\end{Shaded}

Edytujemy plik do pożądanego rezultatu:

\pandocbounded{\includegraphics[keepaspectratio]{/run/media/rafal/ECD5-E984/photos/nano-isc-dhcp-server.png}}

Omówienie konfiguracji:

\begin{Shaded}
\begin{Highlighting}[]
\CommentTok{\# (...) Komentarze (...)}

\CommentTok{\# Podajemy nazwę karty sieciowej LAN\textquotesingle{}owej }
\CommentTok{\# (w moim przypadku enp0s8)}
\DataTypeTok{INTERFACESv4}\OperatorTok{=}\StringTok{"enp0s8"}
\CommentTok{\# Zostawiamy puste }
\DataTypeTok{INTERFACESv6}\OperatorTok{=}\StringTok{""}
\end{Highlighting}
\end{Shaded}

\subsubsection{Krok 2 - Edycja pliku konfiguracyjnego
dhcpd.conf}\label{krok-2---edycja-pliku-konfiguracyjnego-dhcpd.conf}

\begin{Shaded}
\begin{Highlighting}[]
\FunctionTok{sudo}\NormalTok{ nano /etc/dhcp/dhcpd.conf}
\end{Highlighting}
\end{Shaded}

Edytujemy plik do pożądanego rezultatu:

\pandocbounded{\includegraphics[keepaspectratio]{/run/media/rafal/ECD5-E984/photos/nano-dhcpd2.png}}

Omówienie konfiguracji:

\begin{Shaded}
\begin{Highlighting}[]
\CommentTok{\# (...) Komentarze i inne domyślne ustawienia (...)}

\CommentTok{\# Ustawienie serwera DHCP jako jedyny w sieci}
\ExtensionTok{authoritative}\KeywordTok{;}

\CommentTok{\# (...) Komentarze (...)}

\CommentTok{\# subnet {-} adres sieci {-} netmask {-} maska podsieci}
\ExtensionTok{subnet}\NormalTok{ 172.16.0.0 netmask 255.255.0.0 \{}
  \CommentTok{\# Zakres sieci (bez serwera)}
  \CommentTok{\# W moim przypadku 172.16.0.10{-}172.16.0.20}
  \ExtensionTok{range}\NormalTok{ 172.16.0.10 172.16.0.20}\KeywordTok{;}
  \CommentTok{\# Adresy serwerów DNS}
  \ExtensionTok{option}\NormalTok{ domain{-}name{-}servers 172.16.0.1, 1.1.1.1}\KeywordTok{;}
  \CommentTok{\# Nazwa domeny (opcjonalnie)}
  \ExtensionTok{option}\NormalTok{ domain{-}name }\StringTok{"kiepiela.local"}\KeywordTok{;}
  \CommentTok{\# Maska podsieci}
  \ExtensionTok{option}\NormalTok{ subnet{-}mask 255.255.0.0}\KeywordTok{;}
  \CommentTok{\# Adres rozgłoszeniowy}
  \ExtensionTok{option}\NormalTok{ broadcast{-}address 172.16.255.255}\KeywordTok{;}
  \CommentTok{\# Czas zalokowania adresu}
  \ExtensionTok{default{-}lease{-}time}\NormalTok{ 600}\KeywordTok{;}
  \CommentTok{\# Maksymalny czas zalokowania adresu}
  \ExtensionTok{max{-}lease{-}time}\NormalTok{ 7200}\KeywordTok{;}
\ErrorTok{\}}
\end{Highlighting}
\end{Shaded}

\subsubsection{Krok 3 - Zastosowanie konfiguracji oraz sprawdzenie
poprawności wykonanej
konfiguracji}\label{krok-3---zastosowanie-konfiguracji-oraz-sprawdzenie-poprawnoux15bci-wykonanej-konfiguracji-1}

\begin{Shaded}
\begin{Highlighting}[]
\FunctionTok{sudo}\NormalTok{ systemctl restart isc{-}dhcp{-}server}
\end{Highlighting}
\end{Shaded}

\begin{Shaded}
\begin{Highlighting}[]
\FunctionTok{sudo}\NormalTok{ systemctl status isc{-}dhcp{-}server}
\end{Highlighting}
\end{Shaded}

\pandocbounded{\includegraphics[keepaspectratio]{/run/media/rafal/ECD5-E984/photos/systemctl-dhcp.png}}

Jest na zielono? - to oznacza, że wszystko działa :)

\subsection{3. Konfiguracja serwera DNS}\label{konfiguracja-serwera-dns}

\subsubsection{Krok 0 - Instalacja pakietu bind9 i
bind9-utils}\label{krok-0---instalacja-pakietu-bind9-i-bind9-utils}

\begin{Shaded}
\begin{Highlighting}[]
\FunctionTok{sudo}\NormalTok{ apt update}
\end{Highlighting}
\end{Shaded}

\begin{Shaded}
\begin{Highlighting}[]
\FunctionTok{sudo}\NormalTok{ apt install bind9 bind9{-}utils}
\end{Highlighting}
\end{Shaded}

Ja już te pakiety zainstalowałem szybciej, przejdźmy do konfiguracji.

\subsubsection{Krok 1 - Edycja pliku konfiguracyjnego
named.conf.options}\label{krok-1---edycja-pliku-konfiguracyjnego-named.conf.options}

\begin{Shaded}
\begin{Highlighting}[]
\FunctionTok{sudo}\NormalTok{ nano /etc/bind/named.conf.options}
\end{Highlighting}
\end{Shaded}

Edytujemy plik do pożądanego rezultatu:

\pandocbounded{\includegraphics[keepaspectratio]{/run/media/rafal/ECD5-E984/photos/nano-named-conf-options.png}}

Omówienie konfiguracji:

\begin{Shaded}
\begin{Highlighting}[]
\NormalTok{options }\OperatorTok{\{}
  \CommentTok{// Wartość domyślna}
\NormalTok{  directory }\StringTok{"/var/cache/bind"}\OperatorTok{;}

  \CommentTok{// (...) Komendarze (...)}

\NormalTok{  forwarders }\OperatorTok{\{}
    \CommentTok{// Adresy publicznych serwerów DNS}
    \FloatTok{1.1.1.1}\OperatorTok{;}
    \FloatTok{8.8.8.8}\OperatorTok{;}
  \OperatorTok{\}}

  \CommentTok{// (...) Komentarze (...)}

  \CommentTok{// Wartości domyślne}
\NormalTok{  dnssec}\OperatorTok{{-}}\NormalTok{validation auto}\OperatorTok{;}
\NormalTok{  listen}\OperatorTok{{-}}\NormalTok{on}\OperatorTok{{-}}\NormalTok{v6 }\OperatorTok{\{}\NormalTok{any}\OperatorTok{;} \OperatorTok{\};}
\OperatorTok{\}}
\end{Highlighting}
\end{Shaded}

\subsubsection{Krok 2 - Edycja pliku konfiguracyjnego
named.conf.local}\label{krok-2---edycja-pliku-konfiguracyjnego-named.conf.local}

\begin{Shaded}
\begin{Highlighting}[]
\FunctionTok{sudo}\NormalTok{ nano /etc/bind/named.conf.local}
\end{Highlighting}
\end{Shaded}

Edytujemy plik do pożądanego rezultatu:

\pandocbounded{\includegraphics[keepaspectratio]{/run/media/rafal/ECD5-E984/photos/nano-named-conf-local2.png}}

Omówienie konfiguracji:

\begin{Shaded}
\begin{Highlighting}[]
\CommentTok{// (...) Komentarze (...)}

\CommentTok{// Domena (strefa do przodu)}
\NormalTok{zone }\StringTok{"kiepiela.local"}\OperatorTok{\{}
\NormalTok{  type master}\OperatorTok{;}
\NormalTok{  file }\StringTok{"/etc/bind/strefaprzod"}\OperatorTok{;}
\OperatorTok{\};}
\CommentTok{// Adres sieci bez zer od tyłu + ".in{-}addr.arpa" (strefa do tyłu)}
\NormalTok{zone }\StringTok{"16.172.in{-}addr.arpa"}\OperatorTok{\{}
\NormalTok{  type master}\OperatorTok{;}
\NormalTok{  file }\StringTok{"/etc/bind/strefatyl"}\OperatorTok{;}
\OperatorTok{\};}
\end{Highlighting}
\end{Shaded}

\subsubsection{Krok 3 - Utworzenie i edycja pliku konfiguracyjnego
strefaprzod}\label{krok-3---utworzenie-i-edycja-pliku-konfiguracyjnego-strefaprzod}

Kopiowanie domyślnego pliku konfiguracyjnego do naszego:

\begin{Shaded}
\begin{Highlighting}[]
\FunctionTok{sudo}\NormalTok{ cp /etc/bind/db.local /etc/bind/strefaprzod}
\end{Highlighting}
\end{Shaded}

\begin{Shaded}
\begin{Highlighting}[]
\FunctionTok{sudo}\NormalTok{ nano /etc/bind/strefaprzod}
\end{Highlighting}
\end{Shaded}

Edytujemy plik do pożądanego rezultatu:

\pandocbounded{\includegraphics[keepaspectratio]{/run/media/rafal/ECD5-E984/photos/nano-strefaprzod2.png}}

Omówienie konfiguracji:

\begin{Shaded}
\begin{Highlighting}[]
\CommentTok{; (...) komentarze (...)}

\NormalTok{$TTL  }\DecValTok{604800}\CommentTok{;     użytkownik.domena.    root.użytkownik.domena.}
\NormalTok{@     IN  SOA     rafal.kiepiela.local. root.rafal.kiepiela.local.(}
                        \DecValTok{2}         \CommentTok{; Serial}
                   \DecValTok{604800}         \CommentTok{; Refresh}
                    \DecValTok{86400}         \CommentTok{; Retry}
                  \DecValTok{2419200}         \CommentTok{; Expire}
                   \DecValTok{604800}\NormalTok{ )       }\CommentTok{; Negative Cache TTL}
\CommentTok{;}
\CommentTok{;                 użytkownik.domena.}
\NormalTok{@     IN  NS      rafal.kiepiela.local.}
\CommentTok{;                 adres ip serwera}
\NormalTok{@     IN  A       }\DecValTok{172}\NormalTok{.}\DecValTok{16}\NormalTok{.}\FloatTok{0.1}
\CommentTok{;@     IN  AAAA    ::1}
\CommentTok{; użytkownik      adres ip serwera}
\NormalTok{rafal IN  A       }\DecValTok{172}\NormalTok{.}\DecValTok{16}\NormalTok{.}\FloatTok{0.1}
\end{Highlighting}
\end{Shaded}

\subsubsection{Krok 4 - Utworzenie i edycja pliku konfiguracyjnego
strefatyl}\label{krok-4---utworzenie-i-edycja-pliku-konfiguracyjnego-strefatyl}

Kopiowanie pliku konfiguracyjnego strefaprzod do strefatyl:

\begin{Shaded}
\begin{Highlighting}[]
\FunctionTok{sudo}\NormalTok{ cp /etc/bind/strefaprzod /etc/bind/strefatyl}
\end{Highlighting}
\end{Shaded}

\begin{Shaded}
\begin{Highlighting}[]
\FunctionTok{sudo}\NormalTok{ nano /etc/bind/strefatyl}
\end{Highlighting}
\end{Shaded}

Edytujemy plik do pożądanego rezultatu:

\pandocbounded{\includegraphics[keepaspectratio]{/run/media/rafal/ECD5-E984/photos/nano-strefatyl2.png}}

Omówienie konfiguracji:

\begin{Shaded}
\begin{Highlighting}[]
\CommentTok{; (...) komentarze (...)}

\NormalTok{$TTL  }\DecValTok{604800}\CommentTok{;     użytkownik.domena.    root.użytkownik.domena.}
\NormalTok{@     IN  SOA     rafal.kiepiela.local. root.rafal.kiepiela.local.(}
                        \DecValTok{2}         \CommentTok{; Serial}
                   \DecValTok{604800}         \CommentTok{; Refresh}
                    \DecValTok{86400}         \CommentTok{; Retry}
                  \DecValTok{2419200}         \CommentTok{; Expire}
                   \DecValTok{604800}\NormalTok{ )       }\CommentTok{; Negative Cache TTL}
\CommentTok{;}
\CommentTok{;                 użytkownik.domena.}
\NormalTok{@     IN  NS      rafal.kiepiela.local.}
\CommentTok{;                 adres ip serwera}
\NormalTok{@     IN  A       }\DecValTok{172}\NormalTok{.}\DecValTok{16}\NormalTok{.}\FloatTok{0.1}
\CommentTok{;@     IN  AAAA    ::1}
\CommentTok{; użytkownik      adres ip serwera}
\NormalTok{rafal IN  A       }\DecValTok{172}\NormalTok{.}\DecValTok{16}\NormalTok{.}\FloatTok{0.1}

\CommentTok{; nowości od tego momentu:}
\CommentTok{; adres serwera od tyłu bez części sieciowej | użytkownik.domena.}
\FloatTok{1.0}\NormalTok{   IN  PTR     rafal.kiepiela.local.}
\CommentTok{;                 domena.}
\NormalTok{@     IN  PTR     kiepiela.local.}
\end{Highlighting}
\end{Shaded}

\subsubsection{Krok 5 - Zastosowanie konfiguracji oraz sprawdzenie
poprawności wykonanej
konfiguracji}\label{krok-5---zastosowanie-konfiguracji-oraz-sprawdzenie-poprawnoux15bci-wykonanej-konfiguracji}

\begin{Shaded}
\begin{Highlighting}[]
\FunctionTok{sudo}\NormalTok{ systemctl restart bind9}
\end{Highlighting}
\end{Shaded}

\begin{Shaded}
\begin{Highlighting}[]
\FunctionTok{sudo}\NormalTok{ systemctl status bind9}
\end{Highlighting}
\end{Shaded}

\pandocbounded{\includegraphics[keepaspectratio]{/run/media/rafal/ECD5-E984/photos/systemctl-bind.png}}

Jest na zielono? - to oznacza, że wszystko działa :)

\subsubsection{Krok 6 - Edycja pliku konfiguracyjnego
resolv.conf}\label{krok-6---edycja-pliku-konfiguracyjnego-resolv.conf}

\begin{Shaded}
\begin{Highlighting}[]
\FunctionTok{sudo}\NormalTok{ nano /etc/resolv.conf}
\end{Highlighting}
\end{Shaded}

Edytujemy plik do pożądanego rezultatu:

\pandocbounded{\includegraphics[keepaspectratio]{/run/media/rafal/ECD5-E984/photos/nano-resolv.png}}

Omówienie konfiguracji:

\begin{Shaded}
\begin{Highlighting}[]
\CommentTok{\# (...) Komentarze (...)}

\CommentTok{\# Adres serwera {-} na serwerze można ustawić localhost}
\ExtensionTok{nameserver}\NormalTok{ 127.0.0.1}
\CommentTok{\# Domena}
\ExtensionTok{search}\NormalTok{ kiepiela.local}
\end{Highlighting}
\end{Shaded}

Warto też zabezpieczyć plik przed nadpisywaniem przez system podczas
restartu:

\begin{Shaded}
\begin{Highlighting}[]
\FunctionTok{realpath}\NormalTok{ /etc/resolv.conf}
\end{Highlighting}
\end{Shaded}

\begin{Shaded}
\begin{Highlighting}[]
\FunctionTok{sudo}\NormalTok{ chattr +i /run/systemd/resolve/stub{-}resolv.conf}
\end{Highlighting}
\end{Shaded}

\pandocbounded{\includegraphics[keepaspectratio]{/run/media/rafal/ECD5-E984/photos/chattr-resolv.png}}

Aktualizacja systemu jednak może powodować ``zdjęcie'' tej ``ochrony''.

\subsubsection{Krok 7 - Zastosowanie konfiguracji oraz sprawdzenie
poprawności wykonanej konfiguracji
(nslookup)}\label{krok-7---zastosowanie-konfiguracji-oraz-sprawdzenie-poprawnoux15bci-wykonanej-konfiguracji-nslookup}

Warto sprawdzić~czy wszystko działa narzędziem nslookup:

\begin{Shaded}
\begin{Highlighting}[]
\ExtensionTok{nslookup}\NormalTok{ kiepiela.local}
\end{Highlighting}
\end{Shaded}

\begin{Shaded}
\begin{Highlighting}[]
\ExtensionTok{nslookup}\NormalTok{ rafal.kiepiela.local}
\end{Highlighting}
\end{Shaded}

\begin{Shaded}
\begin{Highlighting}[]
\ExtensionTok{nslookup}\NormalTok{ wp.pl}
\end{Highlighting}
\end{Shaded}

\begin{Shaded}
\begin{Highlighting}[]
\ExtensionTok{nslookup}\NormalTok{ 172.16.0.1}
\end{Highlighting}
\end{Shaded}

\begin{Shaded}
\begin{Highlighting}[]
\ExtensionTok{nslookup}\NormalTok{ rafal}
\end{Highlighting}
\end{Shaded}

\pandocbounded{\includegraphics[keepaspectratio]{/run/media/rafal/ECD5-E984/photos/nslookup3.png}}

Jak widać strefa do przodu jak i do tył zwraca pożądany wynik.

\subsubsection{Krok 8 (w razie problemów) - Edycja pliku
konfiguracyjnego
hosts}\label{krok-8-w-razie-problemuxf3w---edycja-pliku-konfiguracyjnego-hosts}

Plik /etc/hosts omija serwery DNS (tak nawet nasz, który
konfigurowaliśmy) i wymusza swoje własne wartości. Takie trochę
rozwiązanie ``na chama''.

\begin{Shaded}
\begin{Highlighting}[]
\FunctionTok{sudo}\NormalTok{ nano /etc/hosts}
\end{Highlighting}
\end{Shaded}

Edytujemy plik do pożądanego rezultatu:

\pandocbounded{\includegraphics[keepaspectratio]{/run/media/rafal/ECD5-E984/photos/nano-hosts.png}}

Omówienie konfiguracji:

\begin{Shaded}
\begin{Highlighting}[]
\ExtensionTok{127.0.0.1}\NormalTok{ localhost}
\ExtensionTok{127.0.1.1}\NormalTok{ kiepiela}
\CommentTok{\# adres serwera | użytkownik}
\ExtensionTok{172.16.0.1}\NormalTok{ rafal}

\CommentTok{\# (...) domyślne ustawienia (...)}
\end{Highlighting}
\end{Shaded}

\subsection{4. Konfiguracja routingu
(NAT)}\label{konfiguracja-routingu-nat}

\subsubsection{Krok 1 - włączenie routingu w systemie (plik
sysctl.conf)}\label{krok-1---wux142ux105czenie-routingu-w-systemie-plik-sysctl.conf}

\begin{Shaded}
\begin{Highlighting}[]
\FunctionTok{sudo}\NormalTok{ nano /etc/sysctl.conf}
\end{Highlighting}
\end{Shaded}

Edytujemy plik do pożądanego rezultatu:

\pandocbounded{\includegraphics[keepaspectratio]{/run/media/rafal/ECD5-E984/photos/nano-sysctl.png}}

Omówienie konfiguracji:

\begin{Shaded}
\begin{Highlighting}[]
\CommentTok{\# Należy odkomentować linijkę poniżej:}
\ExtensionTok{net.ipv4.ip\_forward=1}
\end{Highlighting}
\end{Shaded}

Zastosowanie konfiguracji:

\begin{Shaded}
\begin{Highlighting}[]
\FunctionTok{sudo}\NormalTok{ sysctl }\AttributeTok{{-}p}
\end{Highlighting}
\end{Shaded}

\pandocbounded{\includegraphics[keepaspectratio]{/run/media/rafal/ECD5-E984/photos/sysctl.png}}

\subsubsection{Krok 2 - Konfiguracja maskarady w
firewall'u}\label{krok-2---konfiguracja-maskarady-w-firewallu}

\begin{Shaded}
\begin{Highlighting}[]
                                  \CommentTok{\# nazwa karty WAN}
\FunctionTok{sudo}\NormalTok{ iptables }\AttributeTok{{-}t}\NormalTok{ nat }\AttributeTok{{-}A}\NormalTok{ POSTROUTING }\AttributeTok{{-}o}\NormalTok{ enp0s3 }\AttributeTok{{-}j}\NormalTok{ MASQUERADE}
\end{Highlighting}
\end{Shaded}

\pandocbounded{\includegraphics[keepaspectratio]{/run/media/rafal/ECD5-E984/photos/iptables.png}}

\subsubsection{Krok 3 - Zachowanie routingu po
restarcie}\label{krok-3---zachowanie-routingu-po-restarcie}

Aby zachować ustawienia routingu po restarcie należy zainstalować pakiet
iptables-persistent.

\begin{Shaded}
\begin{Highlighting}[]
\FunctionTok{sudo}\NormalTok{ apt update}
\end{Highlighting}
\end{Shaded}

\begin{Shaded}
\begin{Highlighting}[]
\FunctionTok{sudo}\NormalTok{ apt install iptables{-}persistent}
\end{Highlighting}
\end{Shaded}

Podczas instalacji wyskoczą 2 monity. Należy wybrać ``yes''.

\pandocbounded{\includegraphics[keepaspectratio]{/run/media/rafal/ECD5-E984/photos/iptables-monit-v4.png}}

\pandocbounded{\includegraphics[keepaspectratio]{/run/media/rafal/ECD5-E984/photos/iptables-monit-v6.png}}

Konfigurację warto jeszcze raz ręczenie zapisać:

\begin{Shaded}
\begin{Highlighting}[]
\FunctionTok{sudo}\NormalTok{ iptables{-}save}
\end{Highlighting}
\end{Shaded}

\pandocbounded{\includegraphics[keepaspectratio]{/run/media/rafal/ECD5-E984/photos/iptables-save.png}}

\subsection{5. Serwer plików FTP}\label{serwer-plikuxf3w-ftp}

\subsubsection{Krok 0 - Instalacja pakietu
vsftpd}\label{krok-0---instalacja-pakietu-vsftpd}

\begin{Shaded}
\begin{Highlighting}[]
\FunctionTok{sudo}\NormalTok{ apt update}
\end{Highlighting}
\end{Shaded}

\begin{Shaded}
\begin{Highlighting}[]
\FunctionTok{sudo}\NormalTok{ apt install vsftpd}
\end{Highlighting}
\end{Shaded}

Ja już ten pakiet zainstalowałem szybciej, przejdźmy do konfiguracji.

\subsubsection{Krok 1 - Edycja pliku konfiguracyjnego
vsftpd.conf}\label{krok-1---edycja-pliku-konfiguracyjnego-vsftpd.conf}

Na potrzeby ćwiczeniowe przyjmujemy, że użytkownik anonimowy będzie miał
dostęp do zapisu jak i odczytu.

\begin{Shaded}
\begin{Highlighting}[]
\FunctionTok{sudo}\NormalTok{ nano /etc/vsftpd.conf}
\end{Highlighting}
\end{Shaded}

Edytujemy plik do pożądanego rezultatu:

\pandocbounded{\includegraphics[keepaspectratio]{/run/media/rafal/ECD5-E984/photos/vsftpd.png}}

Omówienie konfiguracji:

\begin{Shaded}
\begin{Highlighting}[]
\CommentTok{\# Włączenie serwera}
\VariableTok{listen}\OperatorTok{=}\NormalTok{YES}
\CommentTok{\# Wyłaczenie nagłaśniania na ipv6 (musi być NO inaczej powoduje problemy)}
\VariableTok{listen\_ipv6}\OperatorTok{=}\NormalTok{NO}
\CommentTok{\# Włączenie dostępu anonimowego}
\VariableTok{anonymous\_enable}\OperatorTok{=}\NormalTok{YES}
\CommentTok{\# Włączenie logowania bez hasła}
\VariableTok{no\_anon\_password}\OperatorTok{=}\NormalTok{YES}
\CommentTok{\# Włączenie zapisu}
\VariableTok{write\_enable}\OperatorTok{=}\NormalTok{YES}
\CommentTok{\# Coś z uprawnieniami}
\VariableTok{local\_umask}\OperatorTok{=}\NormalTok{022}
\VariableTok{anon\_umask}\OperatorTok{=}\NormalTok{022}
\CommentTok{\# Włączenie zapisu dla anonimów}
\VariableTok{anon\_upload\_enable}\OperatorTok{=}\NormalTok{YES}
\CommentTok{\# Włączenie tworzenia folderów dla anonimów}
\VariableTok{anon\_mkdir\_write\_enable}\OperatorTok{=}\NormalTok{YES}
\CommentTok{\# Włączenie innych uprawnień dla anonimów}
\VariableTok{anon\_other\_write\_enable}\OperatorTok{=}\NormalTok{YES}
\CommentTok{\# Ścieżka do folderu (zaraz go utworzymy)}
\VariableTok{anon\_root}\OperatorTok{=}\NormalTok{/srv/anon}

\CommentTok{\# (...) Reszta domyślnie (...)}
\end{Highlighting}
\end{Shaded}

\subsubsection{Krok 2 - Utworzenie odpowiednich folderów i ustawienie
odpowiednich
uprawnień}\label{krok-2---utworzenie-odpowiednich-folderuxf3w-i-ustawienie-odpowiednich-uprawnieux144}

Tworzymy dwa foldery:

\begin{Shaded}
\begin{Highlighting}[]
\FunctionTok{sudo}\NormalTok{ mkdir /srv/anon}
\end{Highlighting}
\end{Shaded}

\begin{Shaded}
\begin{Highlighting}[]
\FunctionTok{sudo}\NormalTok{ mkdir /srv/anon/folder{-}do{-}zapisu}
\end{Highlighting}
\end{Shaded}

I ustawiamy odpowiednie uprawnienia:

\begin{Shaded}
\begin{Highlighting}[]
\FunctionTok{sudo}\NormalTok{ chmod 755 /srv/anon }\CommentTok{\# Tak, ten folder ma mieć uprawnienia zapisu tylko dla właściciela}
\end{Highlighting}
\end{Shaded}

\begin{Shaded}
\begin{Highlighting}[]
\FunctionTok{sudo}\NormalTok{ chmod 777 /srv/anon/folder{-}do{-}zapisu}
\end{Highlighting}
\end{Shaded}

\pandocbounded{\includegraphics[keepaspectratio]{/run/media/rafal/ECD5-E984/photos/vsftpd-folders.png}}

\subsubsection{Krok 3 - Zastosowanie konfiguracji oraz sprawdzenie
poprawności wykonanej
konfiguracji}\label{krok-3---zastosowanie-konfiguracji-oraz-sprawdzenie-poprawnoux15bci-wykonanej-konfiguracji-2}

\begin{Shaded}
\begin{Highlighting}[]
\FunctionTok{sudo}\NormalTok{ systemctl restart vsftpd}
\end{Highlighting}
\end{Shaded}

\begin{Shaded}
\begin{Highlighting}[]
\FunctionTok{sudo}\NormalTok{ systemctl status vsftpd}
\end{Highlighting}
\end{Shaded}

\pandocbounded{\includegraphics[keepaspectratio]{/run/media/rafal/ECD5-E984/photos/systemctl-vsftpd.png}}

Jest na zielono? - to oznacza, że wszystko działa :)

\subsection{6. Serwer plików Samba}\label{serwer-plikuxf3w-samba}

\subsubsection{Krok 0 - Instalacja pakietu
samba}\label{krok-0---instalacja-pakietu-samba}

\begin{Shaded}
\begin{Highlighting}[]
\FunctionTok{sudo}\NormalTok{ apt update}
\end{Highlighting}
\end{Shaded}

\begin{Shaded}
\begin{Highlighting}[]
\FunctionTok{sudo}\NormalTok{ apt install samba}
\end{Highlighting}
\end{Shaded}

Ja już ten pakiet zainstalowałem szybciej, przejdźmy do konfiguracji.

\subsubsection{Krok 1 - Edycja pliku konfiguracyjnego
smb.conf}\label{krok-1---edycja-pliku-konfiguracyjnego-smb.conf}

Na potrzeby ćwiczeniowe przyjmujemy, że użytkownik anonimowy będzie miał
dostęp do zapisu jak i odczytu.

\begin{Shaded}
\begin{Highlighting}[]
\FunctionTok{sudo}\NormalTok{ nano /etc/samba/smb.conf}
\end{Highlighting}
\end{Shaded}

Edytujemy plik do pożądanego rezultatu:

\pandocbounded{\includegraphics[keepaspectratio]{/run/media/rafal/ECD5-E984/photos/nano-smb.png}}

Omówienie konfiguracji:

\begin{Shaded}
\begin{Highlighting}[]
\CommentTok{\# (...) Domyślne ustawienia i kometarze (...)}

\KeywordTok{[anon]}
\DataTypeTok{  }\CommentTok{\# Ścieżka do folderu który zaraz utworzymy}
\DataTypeTok{  path }\OtherTok{=}\StringTok{ /srv/samba}
\DataTypeTok{  }\CommentTok{\# Dostęp anonimowy}
\DataTypeTok{  guest ok }\OtherTok{=}\StringTok{ }\KeywordTok{yes}
\DataTypeTok{  }\CommentTok{\# Przeglądanie zawartości}
\DataTypeTok{  browseable }\OtherTok{=}\StringTok{ }\KeywordTok{yes}
\DataTypeTok{  }\CommentTok{\# Włączenie zapisu}
\DataTypeTok{  read only }\OtherTok{=}\StringTok{ }\KeywordTok{no}
\DataTypeTok{  writeable }\OtherTok{=}\StringTok{ }\KeywordTok{yes}
\DataTypeTok{  }\CommentTok{\# Czy widoczny na liście}
\DataTypeTok{  public }\OtherTok{=}\StringTok{ }\KeywordTok{yes}\StringTok{ }
\DataTypeTok{  }\CommentTok{\# Coś z uprawnieniami}
\DataTypeTok{  create mask }\OtherTok{=}\StringTok{ }\DecValTok{0666}
\DataTypeTok{  directory mask }\OtherTok{=}\StringTok{ }\DecValTok{0777}
\DataTypeTok{  }\CommentTok{\# Wymuszenie danego użytkownika do obsługi serwera }
\DataTypeTok{  }\CommentTok{\# (NIE STOSOWAĆ ROOT\textquotesingle{}a NA PRODUKCJI!)}
\DataTypeTok{  force user }\OtherTok{=}\StringTok{ root}
\end{Highlighting}
\end{Shaded}

\subsubsection{Krok 2 - Utworzenie odpowiednich folderów i ustawienie
odpowiednich
uprawnień}\label{krok-2---utworzenie-odpowiednich-folderuxf3w-i-ustawienie-odpowiednich-uprawnieux144-1}

\begin{Shaded}
\begin{Highlighting}[]
\FunctionTok{sudo}\NormalTok{ mkdir /srv/samba}
\end{Highlighting}
\end{Shaded}

\begin{Shaded}
\begin{Highlighting}[]
\FunctionTok{sudo}\NormalTok{ chmod 777 /srv/samba}
\end{Highlighting}
\end{Shaded}

\pandocbounded{\includegraphics[keepaspectratio]{/run/media/rafal/ECD5-E984/photos/mkdir-samba.png}}

\subsubsection{Krok 3 - Zastosowanie konfiguracji oraz sprawdzenie
poprawności wykonanej
konfiguracji}\label{krok-3---zastosowanie-konfiguracji-oraz-sprawdzenie-poprawnoux15bci-wykonanej-konfiguracji-3}

\begin{Shaded}
\begin{Highlighting}[]
\FunctionTok{sudo}\NormalTok{ systemctl restart smbd}
\end{Highlighting}
\end{Shaded}

\begin{Shaded}
\begin{Highlighting}[]
\FunctionTok{sudo}\NormalTok{ systemctl status smbd}
\end{Highlighting}
\end{Shaded}

\pandocbounded{\includegraphics[keepaspectratio]{/run/media/rafal/ECD5-E984/photos/systemctl-smbd.png}}

Jest na zielono? - to oznacza, że wszystko działa :)

\subsection{7. Konfiguracja serwera WWW}\label{konfiguracja-serwera-www}

\subsubsection{Krok 0.0 - Instalacja pakietu
apache2}\label{krok-0.0---instalacja-pakietu-apache2}

\begin{Shaded}
\begin{Highlighting}[]
\FunctionTok{sudo}\NormalTok{ apt update}
\end{Highlighting}
\end{Shaded}

\begin{Shaded}
\begin{Highlighting}[]
\FunctionTok{sudo}\NormalTok{ apt install apache2}
\end{Highlighting}
\end{Shaded}

Ja już ten pakiet zainstalowałem szybciej, przejdźmy do konfiguracji.

\subsubsection{Krok 0.1 - Założenia}\label{krok-0.1---zaux142oux17cenia}

Scenariusz zakłada, że strona internetowa (plik inny.html jako strona
główna) znajduje się w folderze /srv/www a katalog ma uprawnienie
odczytu dla serwera www (najlepiej 755). Uprawnienia można zmienić w ten
sposób:

\begin{Shaded}
\begin{Highlighting}[]
\FunctionTok{sudo}\NormalTok{ chmod 755 /srv/www}
\end{Highlighting}
\end{Shaded}

\subsubsection{Krok 1 - Konfiguracja pliku konfiguracyjnego
000-default.conf}\label{krok-1---konfiguracja-pliku-konfiguracyjnego-000-default.conf}

\begin{Shaded}
\begin{Highlighting}[]
\FunctionTok{sudo}\NormalTok{ nano /etc/apache2/sites{-}available/000{-}default.conf}
\end{Highlighting}
\end{Shaded}

Edytujemy plik do pożądanego rezultatu:

\pandocbounded{\includegraphics[keepaspectratio]{/run/media/rafal/ECD5-E984/photos/sites.png}}

Omówienie konfiguracji:

\begin{Shaded}
\begin{Highlighting}[]
\OperatorTok{\textless{}}\NormalTok{VirtualHost }\ExtensionTok{*:80}\OperatorTok{\textgreater{}}

   \CommentTok{\# (...) Kometarze (...)}

  \ExtensionTok{ServerAdmin}\NormalTok{ webmaster@localhost }\CommentTok{\# Adres e{-}mail admina (do celów ćwiczeniowych można zostawić to co jest)}
  \ExtensionTok{DocumentRoot}\NormalTok{ /srv/www }\CommentTok{\# ścieżka do folderu ze stroną}

  \CommentTok{\# (...) Kometarze i inne domyślne ustawienia (...)}

  \ExtensionTok{DirectoryIndex}\NormalTok{ inny.html }\CommentTok{\# ustawienie strony głównej na inny.html zamiast domyślnego index.html}
\OperatorTok{\textless{}}\NormalTok{/VirtualHost}\OperatorTok{\textgreater{}}
\end{Highlighting}
\end{Shaded}

\subsubsection{Krok 2 - Konfiguracja pliku konfiguracyjnego
apache2.conf}\label{krok-2---konfiguracja-pliku-konfiguracyjnego-apache2.conf}

\begin{Shaded}
\begin{Highlighting}[]
\FunctionTok{sudo}\NormalTok{ nano /etc/apache2/apache2.conf}
\end{Highlighting}
\end{Shaded}

Edytujemy plik do pożądanego rezultatu:

\pandocbounded{\includegraphics[keepaspectratio]{/run/media/rafal/ECD5-E984/photos/apache2-conf.png}}

Omówienie konfiguracji:

\begin{Shaded}
\begin{Highlighting}[]
\CommentTok{\# (...) Komentarze i inne domyślne ustawienia (...)}

\CommentTok{\#       ścieżka do folderu ze stroną}
\OperatorTok{\textless{}}\NormalTok{Directory }\ExtensionTok{/srv/www/}\OperatorTok{\textgreater{}}
    \CommentTok{\# 3 ustawienia domyślne:}
    \ExtensionTok{Options}\NormalTok{ Indexes FollowSymLinks}
    \ExtensionTok{AllowOverride}\NormalTok{ None}
    \ExtensionTok{Require}\NormalTok{ all granted}
\OperatorTok{\textless{}}\NormalTok{/Directory}\OperatorTok{\textgreater{}}

\CommentTok{\# (...) Komentarze i inne domyślne ustawienia (...)}
\end{Highlighting}
\end{Shaded}

\subsubsection{Krok 3 - Zastosowanie konfiguracji oraz sprawdzenie
poprawności wykonanej
konfiguracji}\label{krok-3---zastosowanie-konfiguracji-oraz-sprawdzenie-poprawnoux15bci-wykonanej-konfiguracji-4}

\begin{Shaded}
\begin{Highlighting}[]
\FunctionTok{sudo}\NormalTok{ systemctl restart apache2}
\end{Highlighting}
\end{Shaded}

\begin{Shaded}
\begin{Highlighting}[]
\FunctionTok{sudo}\NormalTok{ systemctl status apache2}
\end{Highlighting}
\end{Shaded}

Dodatkowo możemy sprawdzić czy strona dotarła by po wpisaniu adresu w
przeglądarce za pomocą narzędzia curl:

\begin{Shaded}
\begin{Highlighting}[]
\ExtensionTok{curl}\NormalTok{ localhost}
\end{Highlighting}
\end{Shaded}

Polecenie powinno zwrócić kod html naszej strony, w tym przypadku pliku
inny.html

\pandocbounded{\includegraphics[keepaspectratio]{/run/media/rafal/ECD5-E984/photos/curl.png}}

Jest na zielono? Curl wyświetla kod strony? - to oznacza, że wszystko
działa :)

\subsection{8. Konfiguracja serwera SSH}\label{konfiguracja-serwera-ssh}

\subsubsection{Krok 0 - Instalacja pakietu
openssh-server}\label{krok-0---instalacja-pakietu-openssh-server}

\begin{Shaded}
\begin{Highlighting}[]
\FunctionTok{sudo}\NormalTok{ apt update}
\end{Highlighting}
\end{Shaded}

\begin{Shaded}
\begin{Highlighting}[]
\FunctionTok{sudo}\NormalTok{ apt install openssh{-}server}
\end{Highlighting}
\end{Shaded}

Ja już ten pakiet zainstalowałem szybciej, przejdźmy do konfiguracji.

\subsubsection{Krok 1 - Uruchomienie usługi
SSH}\label{krok-1---uruchomienie-usux142ugi-ssh}

\begin{Shaded}
\begin{Highlighting}[]
\FunctionTok{sudo}\NormalTok{ systemctl enable ssh }\AttributeTok{{-}{-}now}
\end{Highlighting}
\end{Shaded}

\begin{Shaded}
\begin{Highlighting}[]
\FunctionTok{sudo}\NormalTok{ systemctl status ssh}
\end{Highlighting}
\end{Shaded}

\pandocbounded{\includegraphics[keepaspectratio]{/run/media/rafal/ECD5-E984/photos/ssh.png}}

Jest na zielono? - to oznacza, że wszystko działa :)

\subsection{9. Konfiguracja serwera Telnet (używać SSH, chyba, że
precyzowano, iż~należy
używać~Telneta)}\label{konfiguracja-serwera-telnet-uux17cywaux107-ssh-chyba-ux17ce-precyzowano-iux17c-naleux17cy-uux17cywaux107-telneta}

\subsubsection{Krok 0 - Instalacja pakietu
telnetd}\label{krok-0---instalacja-pakietu-telnetd}

\begin{Shaded}
\begin{Highlighting}[]
\FunctionTok{sudo}\NormalTok{ apt update}
\end{Highlighting}
\end{Shaded}

\begin{Shaded}
\begin{Highlighting}[]
\FunctionTok{sudo}\NormalTok{ apt install telnetd}
\end{Highlighting}
\end{Shaded}

Ja już ten pakiet zainstalowałem szybciej, przejdźmy do konfiguracji.

\subsubsection{Krok 1 - Edycja pliku konfiguracyjnego
inetd.conf}\label{krok-1---edycja-pliku-konfiguracyjnego-inetd.conf}

\begin{Shaded}
\begin{Highlighting}[]
\FunctionTok{sudo}\NormalTok{ nano /etc/inetd.conf}
\end{Highlighting}
\end{Shaded}

Edytujemy plik do pożądanego rezultatu:

\pandocbounded{\includegraphics[keepaspectratio]{/run/media/rafal/ECD5-E984/photos/nano-inetd.png}}

Omówienie konfiguracji:

\begin{Shaded}
\begin{Highlighting}[]
\CommentTok{\# Należy odkomentować linijkę:}
\ExtensionTok{telnet}\NormalTok{  stream  tcp   nowait  root    /usr/sbin/tcpd  /usr/sbin/telnetd}
\end{Highlighting}
\end{Shaded}

\subsubsection{Krok 3 - Zastosowanie konfiguracji oraz sprawdzenie
poprawności wykonanej
konfiguracji}\label{krok-3---zastosowanie-konfiguracji-oraz-sprawdzenie-poprawnoux15bci-wykonanej-konfiguracji-5}

\begin{Shaded}
\begin{Highlighting}[]
\FunctionTok{sudo}\NormalTok{ systemctl restart inetd}
\end{Highlighting}
\end{Shaded}

\begin{Shaded}
\begin{Highlighting}[]
\FunctionTok{sudo}\NormalTok{ systemctl status inetd}
\end{Highlighting}
\end{Shaded}

\pandocbounded{\includegraphics[keepaspectratio]{/run/media/rafal/ECD5-E984/photos/systemctl-inetd.png}}

Jest na zielono? - to oznacza, że wszystko działa :)

\subsection{Bonus - Aktualizacja
systemu}\label{bonus---aktualizacja-systemu}

\subsubsection{Krok 0 - Aktualizacja
repozytoriów}\label{krok-0---aktualizacja-repozytoriuxf3w}

\begin{Shaded}
\begin{Highlighting}[]
\FunctionTok{sudo}\NormalTok{ apt update}
\end{Highlighting}
\end{Shaded}

\pandocbounded{\includegraphics[keepaspectratio]{/run/media/rafal/ECD5-E984/photos/apt-update.png}}

\subsubsection{Krok 1 - Aktualizacja
systemu}\label{krok-1---aktualizacja-systemu}

\begin{Shaded}
\begin{Highlighting}[]
\FunctionTok{sudo}\NormalTok{ apt upgrade}
\end{Highlighting}
\end{Shaded}

\pandocbounded{\includegraphics[keepaspectratio]{/run/media/rafal/ECD5-E984/photos/apt-upgrade.png}}

\begin{Shaded}
\begin{Highlighting}[]
\ExtensionTok{reboot}
\end{Highlighting}
\end{Shaded}

lub

\begin{Shaded}
\begin{Highlighting}[]
\FunctionTok{sudo}\NormalTok{ reboot}
\end{Highlighting}
\end{Shaded}

\subsection{Bonus - Aktualizacja Ubuntu Server do nowszej
wersji}\label{bonus---aktualizacja-ubuntu-server-do-nowszej-wersji}

\subsubsection{Krok 0 - Aktualizacja
repozytoriów}\label{krok-0---aktualizacja-repozytoriuxf3w-1}

\begin{Shaded}
\begin{Highlighting}[]
\FunctionTok{sudo}\NormalTok{ apt update}
\end{Highlighting}
\end{Shaded}

\pandocbounded{\includegraphics[keepaspectratio]{/run/media/rafal/ECD5-E984/photos/apt-update.png}}

\subsubsection{Krok 1 - Aktualizacja Ubuntu Server do nowszej
wersji}\label{krok-1---aktualizacja-ubuntu-server-do-nowszej-wersji}

\begin{Shaded}
\begin{Highlighting}[]
\FunctionTok{sudo}\NormalTok{ apt dist{-}upgrade}
\end{Highlighting}
\end{Shaded}


\end{document}
